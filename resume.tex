\documentclass[11pt]{article}       % set main text size
\usepackage[letterpaper,                % set paper size to letterpaper. change to a4paper for resumes outside of North America
top=0.5in,                          % specify top page margin
bottom=0.5in,                       % specify bottom page margin
left=0.5in,                         % specify left page margin
right=0.5in]{geometry}              % specify right page margin

\usepackage{XCharter}               % set font
\usepackage[T1]{fontenc}            % output encoding
\usepackage[utf8]{inputenc}         % input encoding
\usepackage{enumitem}               % enable lists for bullet points: itemize and \item
\usepackage[hidelinks]{hyperref}    % format hyperlinks
\usepackage{titlesec}               % enable section title customization
\raggedright                        % disable text justification
\pagestyle{empty}                   % disable page numbering

% ensure PDF output will be all-Unicode and machine-readable
\input{glyphtounicode}
\pdfgentounicode=1

% format section headings: bolding, size, white space above and below
\titleformat{\section}{\bfseries\large}{}{-2pt}{}[\vspace{-2pt}\titlerule\vspace{-6.5pt}]

% format bullet points: size, white space above and below, white space between bullets
\renewcommand\labelitemi{$\vcenter{\hbox{\small$\bullet$}}$}
\setlist[itemize]{itemsep=-6pt, leftmargin=12pt, topsep=4pt} %%% Test various topsep values to fix vertical spacing errors

% resume starts here
\begin{document}
	
	% \vspace{-100pt}
	% name
	\vspace{-70pt}
	\centerline{\huge Townsend Southard Pantano}
	
	\vspace{-1pt}
	
	% contact information
	\centerline{\href{mailto:tgsoutha@syr.edu}{tgsoutha@syr.edu} | \href{tel:+19785120472}{+1 (978) 512-0472} | \href{https://github.com/TownsendSP}{github.com/TownsendSP}}
	\vspace{-13.5pt}
	% skills section
	\section*{Skills}
	\textbf{Programming:} C, C++, C\#, Java, SQL, Bash, R, Javascript, Python, Django, PyTorch, Tensorflow, lldb, gdb, x64dbg, CMake, Qt5, Gradle, OpenGL, HTML, \LaTeX, Haskell, Racket\\
	\textbf{Tools:} Microsoft Office, JetBrains IDEs, Binary Ninja, Ghidra, Frida, Burp Suite, Git, Unity, Visual Studio, Radare2
	\textbf{Technical Proficiencies:} Reverse Engineering, Full-Stack, iOS Reversing, Object-Oriented Programming, Machine Learning, Large Language Models, Data Mining \& Analysis, AI, Web Development, Mobile Development, Scripting
	\vspace{-17pt}
	
	% experience section
	\section*{Experience}
	\textbf{Engineering Intern, Agile Cyber Systems} {Assured Information Security} -- Rome, NY \hfill May 2025 -- Current \\
	\vspace{-8pt}
	\begin{itemize}
		\item Architected a provider-agnostic Python framework that enabled rapid integration of LLM capabilities into enterprise products, improving documentation search via RAG and optimized for scalability and efficient inference on laptop-scale platforms.even on consumer GPUs.
		\item Directed project strategy and key technical decisions, leveraging advanced expertise in machine learning and LLM internals to ensure efforts targeted the highest-impact performance and usability improvements.
		\item Optimized system scalability by decoupling embedding from inference, freeing GPU resources for larger models and significantly enhancing response speed, reliability, and accuracy.
		\item Integrated state-of-the-art LLMs (e.g., GLM 4.5 AIR, Qwen-3) within days of release, ensuring clients immediate access to leading-edge capabilities and strengthening product competitiveness.
		\item Orchestrated a CI/CD pipeline using GitLab Actions and Docker and deployed GPU-backed LLM servers to support IDE assistants and integration projects, later enhancing maintainability by migrating to llama.cpp.
	\end{itemize}
	\vspace{-8pt}
	\textbf{Engineering Intern, Analysis \& Exploitation} {Assured Information Security} -- Rome, NY \hfill May 2024 -- May 2025 \\
	\vspace{-8pt}
	\begin{itemize}
		\item Reverse-engineered client binaries to recover Python source code via automated and manual bytecode translation, completing the engagement ahead of schedule and uncovering data retention violations
		\item Assessed iOS app compliance with DHS security standards through reverse engineering, traffic analysis, and model manipulation, uncovering exploitable vulnerabilities in NLP model loading.
		\item Researched DNP3 protocol for a client engagement and developed Python scripts simulating malicious activity.
		\item Authored an iOS security testing handbook detailing workflows for app instrumentation, analysis on jailbroken/non-jailbroken devices, and techniques for identifying new vulnerabilities.
	\end{itemize}
	
	\vspace{-6pt}
	\textbf{Undergraduate Researcher,} {Syracuse University} -- Syracuse, NY \hfill May 2023 -- August 2023 \\
	\vspace{-9pt}
	\begin{itemize}
		\item Contributed to faculty-led research project developing a Windows PE binary corpus compilation system
		\item Implemented PyTorch-based ML workflows, achieving 71\% accuracy in malware detection, validating the dataset's effectiveness for advanced research applications.
		\item Replicated prior binary analysis studies, rewriting code for current disassemblers to validate dataset viability.
		\item Co-authored paper accepted to NeurIPS 2024 (arxiv.org/abs/2405.03991)
	\end{itemize}
	\vspace{-6pt}
	\textbf{ECS Peer Tutor,} {Syracuse University} -- Syracuse, NY \hfill Feb 2023 -- May 2023 \\
	\vspace{-9pt}
	\begin{itemize}
		\item Tutored peers taking the Computer Organization and programming systems course
		\item Helped peers understand the core concepts of the course - x86 Assembly, stack manipulation, memory management, debugging with GDB, and Caches
	\end{itemize}
	
	\vspace{-25pt}
	
	
	% education section
	\section*{Education}
	\textbf{Syracuse University, } {College of Engineering and Computer Science} -- Syracuse, NY \hfill Current GPA: 3.7 \\
	\hspace{2em}BS in Computer Science, \textit{magna cum laude} \hfill May 2025 \\
	\hspace{2em}MS in Computer Science \hfill Anticipated May 2026 \\
	
	\vspace{-20pt}
	% projects section
	\section*{Projects}
	\textbf{Custom Virtualization-Based Obfuscator} \hfill (private repo) \\
	\vspace{-9pt}
	\begin{itemize}
		\item Authored an accompanying research paper and literature review surveying contemporary academic and commercial software-obfuscation techniques.  
		\item Implemented a full ARM RV32i software interpreter in C++ to virtualize targeted functions and provide a controlled execution environment.  
		\item Built a binary-rewriting pipeline using LLVM/Clang and CMake for compilation and Python + Capstone for disassembly/analysis: compile-to-ARM, transform and randomize instruction streams, reassemble, inject obfuscated code as binary sections, and patch call sites to the handler.  
		\item Engineering per-function virtualization primitives (custom instruction-set prototypes, randomized stack layouts) to diversify instruction semantics and raise the cost of static/dynamic analysis - currently in progress.
	\end{itemize}
	\vspace{-7pt}
	\textbf{Secondhand Store Inventory System} \hfill \href{https://github.com/TownsendSP/SecondHandStoreInventorySystem}{github.com/TownsendSP/SecondHandStoreInventorySystem} \\
	\vspace{-9pt}
	\begin{itemize}
		\item Developed a full-stack inventory application using Django, React, Python, and MariaDB
		\item Integrated a multimodal AI model, achieving over $80\%$ accuracy in generating product descriptions and prices
		\item Created a custom internal API for communicating with the AI model to reduce overhead
		\item Self-hosted the AI models and web stack on a GPU-accelerated Linux server
		\item Implemented dynamic image optimization to enhance website performance, improving loading speed 
	\end{itemize}
	\vspace{-7pt}
	
	\textbf{Qt License Verification Program} \hfill \href{https://github.com/TownsendSP/ISC_Crackme}{github.com/TownsendSP/ISC\_Crackme} \\
	\vspace{-9pt}
	\begin{itemize}
		\item Built a cross-platform license verification program in C++ with Qt5 for MacOS, Linux, and Windows
		\item Added internal integrity checks to detect and prevent license tampering
		\item Conducted a workshop reverse engineering the system, including binary patching and license creation using Binary Ninja
		\item Conducted an Information Security club workshop on reverse engineering the system, including binary patching and license creation with Binary Ninja, as well as a detailed write-up to support the workshopwalkthrough document to support the workshopd
	\end{itemize}
	
	\vspace{-7pt}
	
	\textbf{Sentiment Analysis of Social Media Networks} \hfill \href{https://github.com/TownsendSP/Comparative_Sentiment_Analysis}{github.com/TownsendSP/Comparative\_Sentiment\_Analysis} \\
	\vspace{-9pt}
	\begin{itemize}
		\item Analyzed 20M+ Tweets and Bluesky posts from public datasets using sentiment analysis
		\item Optimized ML workloads via parallelization and scheduling to process over 1000 posts/sec on a consumer GPU
		\item Discovered key insights, including more negativity in replies than top-level posts and 20\% more negative posts on Twitter compared to Bluesky
	\end{itemize}
	
	\vspace{-25pt}
	
	\section*{Extracurriculars:}
\textbf{Syracuse Information Security Club} \hfill Sept 2022 – Current \\
\vspace{-9pt}
\begin{itemize}
	\item Qualified for NECCDC Regionals, managing Linux systems, Red Team response, and services.  
	\item \textbf{Treasurer} (May 2024 – May 2025), managing budget and resource allocation to support club operations.  
	\item Transitioned to \textbf{Competition Lead} (May 2025 – Current), designing and leading cybersecurity workshops, coordinating team registration with host schools, and managing travel logistics for in-person competitions.  
	\item Led a team in the Lockdown blue-teaming competition to a $3^{\texttt{rd}}$ place finish.  
\end{itemize}
	
	\vspace{-7pt}
	
	\textbf{Syracuse Club Ski Racing Team} \hfill Sept 2022 - Current\\
	\vspace{-9pt}
	\begin{itemize}
		\item Ski raced competitively in the USCSA collegiate league
	\end{itemize}
\vspace{-25pt}
%	\section*{Notable Courses:}
%	\begin{itemize}
	%		\item \textbf{Evolutionary Machine Learning:} Developed machine learning systems in Python using NumPy
	%		\item \textbf{Computer Graphics:} Used OpenGL and C++ to construct user interfaces and 3D animated scenes
	%		\item \textbf{Pervasive Computing:} Built Android apps with Java and Kotlin
	%		\item \textbf{Virtual Reality:} Made a Virtual Reality application using Unity and C\#
	%		\item \textbf{Virtual Reality:} Made a Virtual Reality aplication using Unity and C\#
	%		\item \textbf{Virtual Reality:} Made a Virtual Reality application using Unity and C\#
	%		\item \textbf{Virtual Reality:} Made a Virtual Reality application using Unity and C\#
	%	\end{itemize}
%

	
	
	\vspace{-18.5pt}
	
	
	
\end{document}

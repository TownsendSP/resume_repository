
% \vspace{-100pt}
% name
\vspace{-70pt}
\centerline{\huge Townsend Southard Pantano}

\vspace{-1pt}

% contact information
\centerline{\href{mailto:tgsoutha@syr.edu}{tgsoutha@syr.edu} | \href{tel:+19785120472}{+1 (978) 512-0472} | \href{https://github.com/TownsendSP}{github.com/TownsendSP}}
\vspace{-13.5pt}
% skills section
\section*{Skills}
\textbf{Programming:} C, C++, C\#, Java, SQL, R, Javascript, Python, Django, 
\SE{CMake, Qt5, Gradle, OpenGL, HTML, \LaTeX, Haskell, Racket}%
\ML{, PyTorch, Tensorflow}\\

\textbf{Tools:} Microsoft Office, JetBrains IDEs
\Cyber{, Ghidra, Frida, Burp Suite, Binary Ninja, IDA}%
\SE{, Git, Unity, Visual Studio, }%
\ML{, Tensorflow, PyTorch}\\

\textbf{Technical Proficiencies:} Reverse Engineering, Full-Stack, Object-Oriented Programming, Web \& Mobile Development, Scripting
\Cyber{, Mobile Reversing, Binary Exploitation and Instrumentation, Web Exploitation, Penetration Testing}%
\ML{, Machine Learning, Large Language Models, Data Mining \& Analysis, AI}
\vspace{-17pt}

% experience section
\section*{Experience}
\textbf{Engineering Intern, Agile Cyber Systems} {Assured Information Security} -- Rome, NY \hfill May 2025 -- Current \\
\vspace{-8pt}
\begin{itemize}
	% General contributions
	\item Designed a provider-agnostic Python framework enabling rapid integration of LLMs into enterprise products, improving documentation search via RAG and optimizing inference on consumer GPUs.
	\item Directed project strategy and key technical decisions to maximize performance, usability, and scalability.
	\item Deployed and maintained GPU-backed LLM servers with CI/CD pipelines using GitLab Actions and Docker.	
	% Variant-specific contributions
	\Cyber{\item Applied secure system design practices to LLM deployment, emphasizing isolation, maintainability, and compliance in enterprise environments.}
	\SE{\item Enhanced framework maintainability by modularizing components and migrating core services to llama.cpp.}
	\ML{\item Integrated state-of-the-art LLMs (GLM 4.5 AIR, Qwen-3) within days of release and optimized GPU memory allocation to support larger, faster models.}
\end{itemize}
\vspace{-8pt}
\textbf{Engineering Intern, Analysis \& Exploitation} {Assured Information Security} -- Rome, NY \hfill May 2024 -- May 2025 \\
\vspace{-8pt}
\begin{itemize}
	\item{Performed penetration testing under a government contract against commercial targets to ensure compliance}
	\Cyber{\item Reverse-engineered client binaries to recover Python source, exposing data-retention violations and delivering results over 2 weeks early.}
	\SE{\item Recovered Python source from obfuscated binaries through automated and manual translation, finishing the engagement over two weeks early and exposing data-retention violations.}
	\ML{\item Reverse-engineered obfuscated Python to expose improper data retention, accelerating completion by 2+ weeks.}	
	\Cyber{\item Assessed iOS app compliance via static and dynamic analysis, finding exploitable flaws in model loading.}
	\SE{\item Exposed NLP model-loading vulnerabilities in iOS apps by manipulating model integration and use}
	\ML{\item Exposed NLP model-loading vulnerabilities in iOS apps by manipulating model integration and traffic flows.}
	\Cyber{\item Modeled DNP3 traffic in Python to support penetration testing and analysis of DNP3 devices and software.}
	\SE{\item Engineered Python tools to replicate DNP3 protocol activity for system testing and validation.}
	\ML{\item Modeled DNP3 traffic in Python to support data-driven penetration testing and analysis of DNP3 traffic.}
	\item Authored an iOS handbook documenting instrumentation, vulnerability discovery, and analysis workflows
	\Cyber{\item Performed instrumentation with Frida on Flutter apps, demonstrating attack vectors compromising user safety.}
	\SE{\item Implemented instrumentation on Flutter apps to validate functionality and uncover architectural weaknesses.}
	\ML{\item Applied instrumentation to ML-enabled mobile apps, demonstrating risks affecting security and reliability.}
\end{itemize}
\vspace{-8pt}
\textbf{Undergraduate Researcher,} {Syracuse University} -- Syracuse, NY \hfill May 2023 -- August 2023 \\
\vspace{-8pt}
\begin{itemize}
	\item Co-authored paper accepted to NeurIPS 2024 (arxiv.org/abs/2405.03991)	
	\Cyber{\item Contributed to faculty research developing a Windows PE binary corpus, supporting large-scale reverse engineering studies.}
	\SE{\item Designed and implemented a corpus compilation system for Windows PE binaries, enabling reproducible software analysis research.}
	\ML{\item Built PyTorch-based malware detection workflows on a Windows PE corpus, achieving 71\% accuracy.}
	\Cyber{\item Recreated prior binary analysis experiments, adapting code for modern disassemblers to validate dataset viability.}
	\SE{\item Replicated and rewrote legacy binary analysis studies, modernizing tooling for current disassemblers and pipelines.}
	\ML{\item Validated dataset quality through replication of prior ML-driven binary analysis studies with updated tools.}
\end{itemize}
\vspace{-8pt}
\textbf{ECS Peer Tutor,} {Syracuse University} -- Syracuse, NY \hfill Feb 2023 -- May 2023 \\
\vspace{-8pt}
\begin{itemize}
	\item Tutored peers taking the Computer Organization and programming systems course
	 \Cyber{\item Guided students in x86 assembly, stack use, memory, GDB debugging, and cache behavior.}
	\SE{\item Taught peers low-level topics: x86 assembly, stack frames, memory, debugging, and cache design.}
	\ML{\item Explained core systems—x86 assembly, memory, debugging, caches as basis for advanced computing.}
\end{itemize}

\vspace{-25pt}


% education section
\section*{Education}
\textbf{Syracuse University, } {College of Engineering and Computer Science} -- Syracuse, NY \hfill Current GPA: 3.7 \\
\hspace{2em}BS in Computer Science, \textit{magna cum laude} \hfill May 2025 \\
\hspace{2em}MS in Computer Science \hfill Anticipated May 2026 \\

\vspace{-20pt}
% projects section
\section*{Projects}
\textbf{Custom Virtualization-Based Obfuscator} \hfill (private repo) \\
\vspace{-9pt}
\begin{itemize}
	\item Authored an accompanying research paper and literature review surveying contemporary academic and commercial software-obfuscation techniques.  
	\item Implemented a full ARM RV32i software interpreter in C++ to virtualize targeted functions and provide a controlled execution environment.  
	\item Built a binary-rewriting pipeline using LLVM/Clang and CMake for compilation and Python + Capstone for disassembly/analysis: compile-to-ARM, transform and randomize instruction streams, reassemble, inject obfuscated code as binary sections, and patch call sites to the handler.  
	\item Engineering per-function virtualization primitives (custom instruction-set prototypes, randomized stack layouts) to diversify instruction semantics and raise the cost of static/dynamic analysis - currently in progress.
\end{itemize}
\vspace{-7pt}
\textbf{Secondhand Store Inventory System} \hfill \href{https://github.com/TownsendSP/SecondHandStoreInventorySystem}{github.com/TownsendSP/SecondHandStoreInventorySystem} \\
\vspace{-9pt}
\begin{itemize}
	\item Developed a full-stack inventory application using Django, React, Python, and MariaDB
	\item Integrated a multimodal AI model, achieving over $80\%$ accuracy in generating product descriptions and prices
	\item Created a custom internal API for communicating with the AI model to reduce overhead
	\item Self-hosted the AI models and web stack on a GPU-accelerated Linux server
	\item Implemented dynamic image optimization to enhance website performance, improving loading speed 
\end{itemize}
\vspace{-7pt}

\textbf{Qt License Verification Program} \hfill \href{https://github.com/TownsendSP/ISC_Crackme}{github.com/TownsendSP/ISC\_Crackme} \\
\vspace{-9pt}
\begin{itemize}
	\item Built a cross-platform license verification program in C++ with Qt5 for MacOS, Linux, and Windows
	\item Added internal integrity checks to detect and prevent license tampering
	\item Conducted a workshop reverse engineering the system, including binary patching and license creation using Binary Ninja
	\item Conducted an Information Security club workshop on reverse engineering the system, including binary patching and license creation with Binary Ninja, as well as a detailed write-up to support the workshopwalkthrough document to support the workshopd
\end{itemize}

\vspace{-7pt}

\textbf{Sentiment Analysis of Social Media Networks} \hfill \href{https://github.com/TownsendSP/Comparative_Sentiment_Analysis}{github.com/TownsendSP/Comparative\_Sentiment\_Analysis} \\
\vspace{-9pt}
\begin{itemize}
	\item Analyzed 20M+ Tweets and Bluesky posts from public datasets using sentiment analysis
	\item Optimized ML workloads via parallelization and scheduling to process over 1000 posts/sec on a consumer GPU
	\item Discovered key insights, including more negativity in replies than top-level posts and 20\% more negative posts on Twitter compared to Bluesky
\end{itemize}

\vspace{-25pt}

\section*{Extracurriculars:}
\textbf{Syracuse Information Security Club} \hfill Sept 2022 – Current \\
\vspace{-9pt}
\begin{itemize}
	\item Qualified for NECCDC Regionals, managing Linux systems, Red Team response, and services.  
	\item \textbf{Treasurer} (May 2024 – May 2025), managing budget and resource allocation to support club operations.  
	\item Transitioned to \textbf{Competition Lead} (May 2025 – Current), designing and leading cybersecurity workshops, coordinating team registration with host schools, and managing travel logistics for in-person competitions.  
	\item Led a team in the Lockdown blue-teaming competition to a $3^{\texttt{rd}}$ place finish.  
\end{itemize}

\vspace{-7pt}

\textbf{Syracuse Club Ski Racing Team} \hfill Sept 2022 - Current\\
\vspace{-9pt}
\begin{itemize}
	\item Ski raced competitively in the USCSA collegiate league
\end{itemize}
\vspace{-25pt}
%	\section*{Notable Courses:}
%	\begin{itemize}
	%		\item \textbf{Evolutionary Machine Learning:} Developed machine learning systems in Python using NumPy
	%		\item \textbf{Computer Graphics:} Used OpenGL and C++ to construct user interfaces and 3D animated scenes
	%		\item \textbf{Pervasive Computing:} Built Android apps with Java and Kotlin
	%		\item \textbf{Virtual Reality:} Made a Virtual Reality application using Unity and C\#
	%		\item \textbf{Virtual Reality:} Made a Virtual Reality aplication using Unity and C\#
	%		\item \textbf{Virtual Reality:} Made a Virtual Reality application using Unity and C\#
	%		\item \textbf{Virtual Reality:} Made a Virtual Reality application using Unity and C\#
	%	\end{itemize}
%



\vspace{-18.5pt}

